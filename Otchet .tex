\documentclass[11pt]{article}
\usepackage{amsmath,amssymb,amsthm}
\usepackage{algorithm}
\usepackage[noend]{algpseudocode} 

%---enable russian----

\usepackage[utf8]{inputenc}
\usepackage[russian]{babel}

% PROBABILITY SYMBOLS
\newcommand*\PROB\Pr 
\DeclareMathOperator*{\EXPECT}{\mathbb{E}}


% Sets, Rngs, ets 
\newcommand{\N}{{{\mathbb N}}}
\newcommand{\Z}{{{\mathbb Z}}}
\newcommand{\R}{{{\mathbb R}}}
\newcommand{\Zp}{\ints_p} % Integers modulo p
\newcommand{\Zq}{\ints_q} % Integers modulo q
\newcommand{\Zn}{\ints_N} % Integers modulo N

% Landau 
\newcommand{\bigO}{\mathcal{O}}
\newcommand*{\OLandau}{\bigO}
\newcommand*{\WLandau}{\Omega}
\newcommand*{\xOLandau}{\widetilde{\OLandau}}
\newcommand*{\xWLandau}{\widetilde{\WLandau}}
\newcommand*{\TLandau}{\Theta}
\newcommand*{\xTLandau}{\widetilde{\TLandau}}
\newcommand{\smallo}{o} %technically, an omicron
\newcommand{\softO}{\widetilde{\bigO}}
\newcommand{\wLandau}{\omega}
\newcommand{\negl}{\mathrm{negl}} 

% Misc
\newcommand{\eps}{\varepsilon}
\newcommand{\inprod}[1]{\left\langle #1 \right\rangle}

 
\newcommand{\handout}[5]{
  \noindent
  \begin{center}
  \framebox{
    \vbox{
      \hbox to 5.78in { {\bf Научно-исследовательская практика} \hfill #2 }
      \vspace{4mm}
      \hbox to 5.78in { {\Large \hfill #5  \hfill} }
      \vspace{2mm}
      \hbox to 5.78in { {\em #3 \hfill #4} }
    }
  }
  \end{center}
  \vspace*{4mm}
}

\newcommand{\lecture}[4]{\handout{#1}{#2}{#3}{#4}{Бинарный алгоритм нахождения НОД #1}}

\newtheorem{theorem}{Теорема}
\newtheorem{lemma}{Лемма}
\newtheorem{definition}{Определение}
\newtheorem{corollary}{Следствие}
\newtheorem{fact}{Факт}

% 1-inch margins
\topmargin 0pt
\advance \topmargin by -\headheight
\advance \topmargin by -\headsep
\textheight 8.9in
\oddsidemargin 0pt
\evensidemargin \oddsidemargin
\marginparwidth 0.5in
\textwidth 6.5in

\parindent 0in
\parskip 1.5ex

\begin{document}

\lecture{}{Лето 2020}{}{Завертанов Владислав}

\section{Описание}
\textbf{Бинарный алгоритм нахождения НОД} —метод нахождения наибольшего общего делителя двух целых чисел. Данный алгоритм быстрее обычного алгоритма Евклида, т.к. вместо медленных операций деления и умножения используются сдвиги. Возможно, алгоритм был известен еще в Китае 1-го века, но опубликован был лишь в 1967 году израильским физиком и программистом Джозефом Стайном.Он основан на использовании  свойств НОД,а именно НОД$(2m, 2n)$ = 2 НОД$(m, n)$, НОД$(2m, 2n+1)$ = НОД$(m, 2n+1)$, НОД$(-m, n)$ = НОД$(m, n)$.


\section{Алгоритм}

\begin{algorithm}[ph]
	\caption{Бинарный алгоритм Евклида}
	\label{alg:AlgName}
	\textbf{Ввод:} a>0, b>0  \\
	\textbf{Вывод:} (a, b)
	
	\begin{algorithmic}[1]
		
		\State представить \textbf{a} и \textbf{b} в виде $ a = 2^ia_1 $,$ b = 2^jb_1 $, где $ a_1, b_1 $ - нечетные
		\State положить $a = a_1, b = b_1$, и найти $ k = min(i, j)$                    
		\State пока \textbf{a} и \textbf{b} не равны, выполняй: 
			\State \textbf{if} $a < b$  \textbf{then} поменять $a$ и $b$ местами
			\State вычислить $c = a - b$ и представить в виде $c = 2^sc_1$, где $c_1$-нечетное
		           \State положить $a = c_1$   
		\State \Return $2^ka$
	\end{algorithmic}

\end{algorithm}
\clearpage

\section{Анализ}

\subsection{Описание}
Алгоритм был реализован на \textit{Sage} — системе компьютерной алгебры, покрывающей много областей математики, включая алгебру, комбинаторику, вычислительную математику и матанализ. Вся вычислительная нагрузка приходилась на облачный ресурс Collaborative Calculation and Data Science (Cocalc).

\subsection{Сравнение результатов}

Для проверки корректности реализации алгоритма была использована встроенная в \textit{Sage} функция \textit{gcd(a, b)}. Были зафиксированы результаты работы и среднее время их исполнения.

\begin{center}

\begin{tabular}{|c|c|c|c|c|c|}
\hline
\multicolumn{2}{|c|}{Параметры} & \multicolumn{2}{c|}{Собственная реализация} & \multicolumn{2}{c|}{Sage} \\ \hline
      a    &      b     &    Результат       &     Время, с      &      Результат     &    Время, с       \\ \hline
      $321^{245}$     &     $768^{82}$      &     1330...7449678409      &     0.00      &     1330...7449678409      &    0.00       \\ \hline
       $12304^{24322}$    &     $8246^{15633}$      &     10044358...951726592     &     0.01      &     10044358...951726592     &     0.01      \\ \hline
       $1230424^{243222}$    &      $824823^{15633}$     &     1      &   0.07        &     1      &        0.06    \\ \hline
       $123456789^{24322} $    &     $77777777777^{15633}$      &     1     &     0.11      &     1     &     0.09      \\ \hline
\end{tabular}
\end{center}


\end{document}